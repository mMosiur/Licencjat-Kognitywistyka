\chapter{Neuronaukowe korzenie uczenia maszynowego}

Już niedługo po pierwszych teoriach i propozycjach dotyczących maszyn przeprowadzających obliczenia przedstawiły się one jako ciekawa koncepcja dla filozofii umysłu i psychologii.
Maszyna Turinga ze swoimi możliwościami przetwarzania informacji oraz terminologia z nią związana stały się nową popularną metaforą działania umysłu.
Aktualnie prawie każdy większy dział nauk humanistycznych ma swój komputacjonistyczny odłam skupiający się na tym właśnie aspekcie jako wbudowanej cesze świata rzeczywistego.
Był to kolejny z trendów analogii i metafory w psychologii \cite{vroon1987man}, jednak z biegiem czasu przez wielu uważany za coraz bliższy prawdy.

Największą różnicą między możliwościami umysłu człowieka a tym, co w przyszłości zostanie nazwane komputerem była koncepcja uczenia się.
Pierwsze użycie sformułowania "uczenie maszynowe" przypisuje się Arturowi Samuelowi, który takim określeniem opisał algorytmy podejmujące decyzje bez wcześniejszego jawnego zaprogramowania ich do takiego ich wykonywania \cite{koza1996automated}.
Dzisiaj ten dział informatyki obudowuje wiele różnych paradygmatów takich algorytmów.
Najwcześniej używanym były algorytmy genetyczne inspirowane ewolucyjnym charakterem przyrody.

Tak jak psychologia czerpała z informatyki inspirację przy budowaniu metafory komputerowej, tak dziedzina uczenia maszynowego inspirowała się obecnymi w naturze zjawiskami wszelkiego rodzaju uczenia się.
Idea algorytmów genetycznych i programowania genetycznego została wyprowadzona jako przeniesienie teorii ewolucji w naturze.
Najbardziej fundamentalna jednostka popularnych dzisiaj sztucznych sieci neuronowych -- sztuczny neuron, czy perceptron -- zostały po raz pierwszy zaproponowane jako matematyczny model naśladujący działanie neuronu biologicznego.

