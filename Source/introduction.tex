\chapter*{Wstęp}
\addcontentsline{toc}{chapter}{Wstęp}

Alan Turing był prawdopodobnie pierwszą osobą, która proponowała zbudowanie sztucznego systemu obliczeniowego przy pomocy prostych jednostek przypominających neurony zorganizowanych w strukturę sieciową \cite{turing1969intelligent}.
Nazwał je ``maszynami niezorganizowanymi''.
Dzisiaj bardzo dobrze znany jest podobny koncept -- sztuczne sieci neuronowe -- na których opiera się dziedzina uczenia głębokiego, podzbiór technik uczenia maszynowego odznaczający się największymi osiągnięciami.

Jego początki były mocno inspirowane działaniem biologicznego mózgu i próbą jego uproszczonego modelowania.
Z czasem obie dziedziny -- uczenie maszynowe oraz neuronauka -- zaczęły się rozjeżdżać.
Pierwsza skupiła się na optymalizacji funkcji kosztu \cite{sutskever2013importance}, gdzie w drugiej pojawiły się istotne aspekty strukturalne, nowe typy komórek czy zachodzących procesów \cite{solari2011cognitive}.
Wraz z czasem różnice się tylko pogłębiały i ludzie przestali traktować tw dziedziny jako powiązane czy posiadające podobieństwa strukturalne.

Nowoczesne badania, odkrycia po stronie neuronauki oraz nowe rozwiązania stosowane w uczeniu głębokim pozwalają jednak sądzić, że ten trend się odwraca.
Wysokopoziomowe struktury mózgu funkcjonalnie przypominają struktury wykorzystywane w uczeniu maszynowym, a z kolei ono zaczyna wykorzystywać techniki i rozwiązania, które można zaobserwować w działaniu mózgu.

Należy tutaj zaznaczyć, że mowa jest o \emph{mózgu} -- biologicznym narządzie opowiadającym za działania poznawcze.
Często dyskusja dotycząca sztucznej inteligencji i jej możliwości jest przeciwstawiana \emph{umysłowi}.
W niektórych aspektach, najczęściej psychologicznych, spekulacje odnośnie umysłu są dość bliskie fizycznego systemom funkcjonalnym w mózgu, takim jak pamięć, systemy uwagowe i inne podobne.
Pojawiają się natomiast także aspekty filozoficzne, a przede wszystkim świadomość czy intencjonalność działania, które zawsze towarzyszą dyskusji na temat sztucznej inteligencji a umysłu.
Pojawiają się głosy mocno sceptyczne \cite{searle1980minds}, jak i przeciwnie, gdzie zaznacza się znaczenie neuronauki w tematyce świadomości \cite{reggia2013rise}.

W tej pracy jednak aspekty fenomenologiczne zostaną pominięte.
Przede wszystkim przedmiotem zestawienia mają być funkcjonalne działanie mózgu wspierane biologicznymi strukturami je opisującymi lub umożliwiającymi, a ogólnymi założeniami uczenia maszynowego oraz technikami i strukturami używanymi w uczeniu głębokim.
Analizując nowe badaniami i techniki można zauważyć, że te dwie koncepcje zaczynają siebie wzajemnie bardzo przypominać.

Potwierdzenie takich powiązań może być bardzo owocne zarówno dla uczenia głębokiego, w konsekwencji maszynowego a także neuronauki.
Niepodważalnie mózg jest niesamowicie efektywny i wydajny w tym, co robi.
Czy to jest przetwarzanie symboliczne informacji o świecie w sposób zbliżony do tego, jak robi to komputer jest nadal debatowane, jednak mocno podparte -- oraz w tej pracy także przedstawiane.
Jeśli tak, to wykorzystanie technik obecnych w mózgu może być dla uczenia maszynowego kolejnym, wielkim skokiem w przód.
Co więcej, analiza działania takich systemów -- nieporównywalnie prostsza w wykonaniu niż analiza działania mózgu -- może poszerzyć wiedzę neuronaukową w sposób dotychczas niedostępny.

W celu wyciągania takich wniosków należy najpierw sprawdzić założenia, z których miałyby wynikać.
Takie powiązania wymagałoby od mózgu oraz systemów uczenia maszynowego strukturalnego powiązane czy podobieństwa funkcjonalnego w budowie oraz aspektów takich, jak czy mózg można traktować jako system symboliczny.
Dalej należy pokazać możliwe powiązania w samym działaniu, ich zasadność i wspierające je badania.
Celem tej pracy jest przedstawienie aktualnego stanu wiedzy świata naukowego na te tematy, przedstawienie opinii i działań badaczy opisujących wspomniane problemy, zbierając w jedno miejsce ich rezultaty oraz wyniki.
Jak się okazuje, wspólnie przedstawiają one optymistyczny obraz potencjalnego ponownej integracji dziedzin uczenia maszynowego oraz neuronauki.
