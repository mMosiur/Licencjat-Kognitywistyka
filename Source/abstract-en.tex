\begin{abstract-en}
    From the very beginning of the field, machine learning has drawn its inspiration from nature: evolutionary algorithms, self-organizing networks, and many more.
    The use of ideas from neurobiology in turn gave birth to artificial neural networks and deep learning.
    The initial hope was that the two fields would support and push each other forward.
    However, over time, the trend has proven to be the opposite, with deep learning and neuroscience being developed in opposite directions.

    Recent discoveries and techniques in neuroscience and deep learning, however, may lead one to believe that this trend has reversed and that these fields are heading back towards each other.
    Reconnecting the two fields would mean a much simpler way for neuroscience to study the workings of the brain by comparing it with its counterpart in deep learning.
    For the latter that would probably mean another leap in the scope of what's possible -- modeling the workings of this evolutionarily shaped and incredibly efficient organ may offer a way to solve current machine learning problems.

    In this work, I present the results of a priming study on the structural comparison of the brain and modern artificial neural networks.
    Additionally, I put together a collection of scientifically supported opinions and hypotheses from researchers on the nature of the brain, deep learning, and how they are related.
\end{abstract-en}