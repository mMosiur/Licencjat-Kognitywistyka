\chapter{Neuronaukowe korzenie uczenia maszynowego}
\label{chapter1}

Już niedługo po pierwszych teoriach i propozycjach dotyczących maszyn przeprowadzających obliczenia przedstawiły się one jako ciekawa koncepcja dla filozofii umysłu i psychologii.
Maszyna Turinga ze swoimi możliwościami przetwarzania informacji oraz terminologia z nią związana stały się nową popularną metaforą działania umysłu.
Aktualnie prawie każdy większy dział nauk humanistycznych ma swój komputacjonistyczny odłam skupiający się na tym właśnie aspekcie jako wbudowanej cesze świata rzeczywistego.
Był to kolejny z wielu innych trendów analogii i metafory w psychologii \cite{vroon1987man}, jednak z biegiem czasu przez wielu uważany za najbliższy prawdy.

\section{Algorytmy uczenia maszynowego inspirowane naturą i biologią}

Największą różnicą między możliwościami umysłu człowieka a tym, co w przyszłości zostanie nazwane komputerem była koncepcja uczenia się.
Pierwsze użycie sformułowania "uczenie maszynowe" przypisuje się Arturowi Samuelowi, który takim określeniem opisał algorytmy podejmujące decyzje bez wcześniejszego jawnego zaprogramowania ich do takiego ich wykonywania \cite{koza1996automated}.
Dzisiaj dział informatyki, nazwany identycznie jak wspomniana grupa algorytmów, skupia w sobie wiele różnych paradygmatów i podejść, z których każde ma własne założenia, ograniczenia i osiągnięcia.
Jednymi z najwcześniej używanych były populacyjno-stochastyczne algorytmy genetyczne, inspirowane ewolucyjnym charakterem przyrody.

Tak jak psychologia czerpała z informatyki inspirację przy budowaniu metafory komputerowej, tak dziedzina uczenia maszynowego inspirowała się obecnymi w naturze zjawiskami wszelkiego rodzaju uczenia się.
Idea algorytmów genetycznych i programowania genetycznego została wyprowadzona jako bezpośrednie przeniesienie teorii ewolucji w naturze.
Jest to jednocześnie przykład, jak uczenie maszynowe może poprzez mechanizm sprzężenia zwrotnego wspomóc obszar jego pierwotnej inspiracji.
Użycie algorytmów genetycznych było pomocne przy demonstracjach i dowodzeniu teorii z zakresu Darwinowskiej ewolucji, takich jak wpływ adaptacji na selekcję, i zwrotny wpływ selekcji na adaptację \cite{bruderer1996organizational}.

Najbardziej fundamentalna jednostka popularnych dzisiaj sztucznych sieci neuronowych -- sztuczny neuron, czy perceptron -- zostały po raz pierwszy zaproponowane jako matematyczny model naśladujący działanie neuronu biologicznego \cite{mcculloch1943logical}.
Początkowo neuron McCullocha-Pittsa bezpośrednio odwzorowywał działanie biologicznego odpowiednika, w tym na przykład traktując wyjście modelu binarnie jako \emph{aktywne} lub \emph{nieaktywne}.
Później także podobnie jak w biologicznym układzie nerwowym złożonym ze skomplikowanej sieci neuronów zaczęto wykorzystywać perceptrony wielowarstwowe propagujące sygnał wgłąb modelu \cite{rosenblatt1961principles}.
Obszar uczenia maszynowego skupiający się właśnie na wykorzystaniu takich wielowarstwowych sztucznych sieci neuronowych został nazwany \emph{uczeniem głębokim} (\emph{deep learning}), podkreślając jednocześnie rolę ilości warstw w skuteczności działania takich sieci.

Sukces nadzorowanego głębokiego uczenia może wkrótce zostać prześcignięty przez \emph{uczenie przez wzmacnianie} (\emph{reinforcement learning}), które również inspirowane jest konceptami neuronaukowymi.
W tym podejściu zamiast uczenia systemu na bazie ogromnej ilości opisanych i skategoryzowanych danych pozwalamy mu na iteracyjne poprawianie swojego działania poprzez odpowiedź środowiska, w którym działa czy nawet własnego klona.
Przypominać to może sposób uczenia się organizmów żywych.

Warto tutaj również sprecyzować użytą wyżej terminologią.
Uczenie przez wzmacnianie również może być skategoryzowane jako podzbiór głębokiego uczenia -- opiera się ono bowiem na głębokich, wielowarstwowych sieciach neuronowych, podpadając pod odpowiednią definicję.
Rozróżnienie to jest tutaj wynikiem dość częstego nazywania nadzorowanego głębokiego uczenia maszynowego po prostu uczeniem głębokim.
Uczenie nadzorowane (\emph{supervised learning}) jest jedną z trzech najpopularniejszych podzbiorów uczenia maszynowego obok uczenia nienadzorowanego (\emph{unsupervised learning}) oraz właśnie uczenia przez wzmacnianie.
Podział ten opierany jest na sposobie uczenia się, gdzie określenie \emph{uczenie głębokie} odnosi się do rodzaju użytych sztucznych sieci neuronowych i zahacza o wszystkie z tych grup algorytmów.

Jak widać na podanych przykładach, wiele systemów oraz idei obszaru uczenia maszynowego miało swoją inspirację w naturze czy neurobiologii.
Dodatkowo od strony neuronauki pokładano nadzieje, że rozwój tych sztucznych systemów wzorowanych na biologicznym układzie nerwowym pozwoli nam w lepszy sposób zrozumieć działanie mózgu, podobnie jak w przypadku algorytmów genetycznych.
Znacznie prostsze jest zrozumienie złożonego systemu poprzez zbudowanie go od podstaw niż próby wykorzystania inżynierii wstecznej na danych obserwacyjnych z nim związanych \cite{braitenberg1986vehicles}.

Obie dziedziny zaczęły się jednak z czasem coraz bardziej oddalać od siebie oddalać, przeciwnie do spekulacji.

Rozwój uczenia maszynowego -- w tym przede wszystkim uczenia głębokiego -- opierał się na optymalizacji funkcji kosztu i odkryciach natury matematycznej z tym związanych \cite{sutskever2013importance}.
Mimo tego, że koncepcje podstawowe wykorzystane w sztucznych sieciach neuronowych były bezpośrednio zainspirowane biologicznymi odpowiednikami, szybko okazało się, że wprowadzane modyfikacje pierwotnej architektury i działania sieci znacznie poprawiają ich skuteczność.
Zmiany te jednak nie tylko nie były wynikiem przełożenia wiedzy neuronaukowej do uczenia maszynowego, a wręcz przeciwnie -- odstępowały od podobieństwa do natury na rzecz złożonych modeli matematycznych oraz udogodnień i usprawnień z nich wynikających.
Problem funkcji kosztu okazał się wyjątkowo ważny w sieciach głębokich, a do jego rozwiązania zaczęto wykorzystywać algorytm wstecznej propagacji błędu \cite{rumelhart1985learning}.
Jego skuteczność zależała natomiast od wykorzystanej funkcji aktywacji.
Funkcja binarna funkcja -- podobna do zachowania neuronu biologicznego -- okazała się gorsza od funkcji liniowych \cite{minsky2017perceptrons}, a te znowu ustępowały funkcjom nieliniowym \cite{haykin1994neural}.
Nowe wykorzystywane funkcje były skuteczniejsze i efektywniejsze \cite{sharma2017activation} w rozwiązywaniu problemów poszukiwania minimum funkcji kosztu.
Jednocześnie oddalały się od swoich biologicznych korzeni działania.

Z drugiej strony w neuronauce odkryto wiele typów komórek, stanów komórkowych, mechanizmów obliczeniowych i przechowywania informacji oraz obszarów mózgowych \cite{solari2011cognitive}.
Nacisk kładziony był raczej na strukturę i budowę mózgu oraz różnej skali elementów go budujących, a sieci neuronowe zaczęły przyjmować raczej rolę zwykłego narzędzia jak w każdej innej dziedzinie, bez większych, mocniejszych czy bezpośrednich powiązań.
Szukano w mózgu podobnych technik działania jak te implementowane w sztucznych sieciach neuronowych, jednak okazywały się one w zdecydowanej większości nieistniejące.
Wcześniej wspomniana wsteczna propagacja błędu nie ma swojego odpowiednika w mózgu \cite{hassabis2017neuroscience}, nawet takiego, który byłby tylko zbliżony ideowo.

Warto również wspomnieć, że rozwój obszaru uczenia maszynowego nie wpłynął na dziedziny odnoszące się do komputerów jako systemów symbolicznych i ich zestawienia z człowiekiem.
W psychologii nadal dominowała i dominuje metafora komputerowa \cite{casey1989computational}, nie przeniesiono uwagi na sztuczne sieci neuronowe, mimo, że wydają się one z pozoru być lepszym odzwierciedleniem mózgu człowieka, głównego źródła poznania i umysłu.
Podobnie w filozofii -- nawet, gdy sieci neuronowe były już znane, nadal odwoływano się do komputera jako całości sprowadzając uczenie maszynowe do wyłącznie algorytmu wykonywanego przez urządzenie symboliczne \cite{searle1980minds}.
Ogólne postrzeganie tej dziedziny sztucznej inteligencji nie wydawało się więc być bliżej działania mózgu -- a w konsekwencji człowieka -- niż tradycyjny pogląd na komputer.

\section{Rozbieżności między uczeniem głębokim a funkcjonowaniem mózgu}
\label{sec:differences}

Oczywiste jest, że mózg, jako byt architekturalnie, logicznie i fizycznie różny od uczenia maszynowego czy szerzej pojętych komputerów, będzie działał inaczej niż one.
Nawet porównując ze sobą bezpośrednio mózgi dwóch osób okazuje się, że czasem są one nie tylko funkcjonalnie, ale także strukturalnie różne.
Różnią się mózgi mężczyzn i kobiet \cite{ingalhalikar2014sex}, różnić się będzie mózg psychopaty \cite{fallon2017neuroanatomical} czy nawet różnić się będą mózgi osób z różnych klas społecznych \cite{rushton1996brain}.
Kiedy więc nawet pozornie identyczne obiekty zaczynają wykazywać spore różnice strukturalne, nie jest zaskoczeniem, że fundamentalnie inne modele przetwarzania informacji będą działać sprzecznie z działaniem ludzkiego mózgu.

Wszelkie zgodności pomiędzy poszczególnymi metodami głębokiego uczenia maszynowego i działaniem mózgu nie muszą, a nawet nie mogą być generalizowane do całości dziedziny \cite{de2018deep}.
Warto jednak przeanalizować ten aspekt, gdyż te fundamentalnie różne modele były pierwotnie inspirowane oraz bazowane na neuronauce \cite{hassabis2017neuroscience}.
Powstałe w czasie różnice nie były przypadkowe, ale kierowane chęcią optymalizacji i specjalizacji algorytmów.
Być może te udoskonalenia algorytmów -- oryginalnie neuro-podobnych -- będą w stanie powiedzieć nam coś o naszych własnych mózgach.

Jeden z przełomów uczenia głębokiego opierał się na tym, że sieci wielowarstwowe uczono stopniowo -- warstwa po warstwie, czyli w procesie jaki ciężko sobie wyobrazić zachodzący w mózgu \cite{hinton2006fast}.
W szczególności w uczeniu głębokim, w zasadzie z samej definicji dziedziny, sieci posiadają znacznie więcej warstw niż odpowiadające im systemy mózgowe -- dziesiątki, często setki.
Inny niedawny przełom, jeden z głównych czynników stojących za sukcesami znanego projektu AlphaGo Zero, to sieci rezydualne.
Ta metoda uczenia głębokiego opiera się na bezpośrednim połączeniu neuronów warstw niższych z neuronami w warstwach wyższych na zasadzie skrótu przeskakując wszystkie warstwy pomiędzy.
Takie struktury nie odzwierciedlają rzeczywistych, biologicznych systemów neuronalnych.
Sieci rezydualne byłyby odpowiednikiem bezpośredniego przekazywania sygnału z pierwszorzędowej kory wzrokowej (V1) do obszaru wzrokowego (V4) czy dolnej kory skroniowej (IT), co jest sprzeczne z rzeczywistym działaniem kory wzrokowej.

Uczenie nadzorowane nadal wymaga ogromnych ilości danych, aby mogło rzeczywiście osiągnąć pozytywne rezultaty.
Mózg działa podobnie, jednak wymaga nieporównywalnie mniej materiału do nauki aby wyciągnąć odpowiednie wnioski -- jest znacznie lepszy w generalizacji przy niewielkiej ilości danych.
Gdyby dziedzina Big Data nie rozwinęła się w takim stopniu, w jakim jest aktualnie dostępna, uczenie głębokie nigdy nie miałoby tak dobrych wyników, jakie osiąga.
Wraz z rozwojem uczenia maszynowego ta różnica potrzebnego materiału nie tylko nie została zmniejszona, ale przeciwnie -- nowsze, lepsze i potężniejsze modele są zwykle uczone na stale rosnącej liczbie danych.
Najnowszy model językowy GPT-3 od OpenAI trenowany był na 499 miliardach tokenów \cite{brown2020language}, co w zależności od średniej wielkości tokenu (4 - 5 bajtów)\footnotemark daje od \(1.8\) do \(2.3\) terabajtów danych czysto tekstowych.
\footnotetext{
	Średnia długość słowa w języku angielskim wynosi nieco ponad 5 znaków \cite{bochkarev6109average}.
	Biorąc pod uwagę, że tokenizacja danych tekstowych w zastosowaniach do przetwarzania języka naturalnego zazwyczaj się opiera na słowach, można założyć, że średnia długość tokenu będzie takiej samej długości.
	Inni autorzy szacują jednak średnią długość tokenu na 4 znaki \cite{li2020OpenAI}.
}

