\chapter{Dyskusja}

Porównywanie człowieka i~sztucznej inteligencji zazwyczaj sprowadza się do zestawienia uczenia maszynowego i~\emph{umysłu} oraz jego właściwości.
W szczególności dyskusje sprowadzają się do świadomości: część badaczy uważa, że sztuczne systemy nie mogą posiadać świadomości \cite{zlatev2000meaning}, inni z~kolei przeciwnie \cite{dietrich2001banbury, reggia2013rise, mallakin2019integration}.

Jednak w~tej pracy dyskusja dotyczy innego zestawienia: uczenia maszynowego i~\emph{mózgu}.
W tym wypadku głosy nie są tak podzielone, a~zdecydowana większość badaczy, że działanie jednego i~drugiego w~pewnym stopniu są zbliżone czy podobne.
Daniel Yamins wykazał, że system uczenia głębokiego w~procesie identyfikacji obiektów na obrazku robił to nie tylko prawie tak dobrze jak człowiek, ale również w~sposób blisko naśladujący proces przetwarzania wizji w~mózgu  \cite{yamins2014performance}.
Podobne obserwacja pojawiły się także w~innych miejscach: działanie systemu słuchowego człowieka ponownie przypomina działanie sztucznych sieci neuronowych trenowanych do tego zadania \cite{kell2018task}, analogicznie sytuacja wyglądała przy analizie układów przetwarzania ruchu \cite{lillicrap2013preference}.

Nie jest więc bezpodstawne sądzić, że rzeczywiście jakieś ścisłe powiązanie między tymi dwiema dziedzinami istnieje.
Zasadniczą różnicą jest oczywiście sama budowa -- sztuczna sieć neuronowa nie posiada biologicznych neuronów, a~neurony w~mózgu nie są funkcjami.
Mimo to analiza budowy i~struktury funkcjonalnej ponownie sugeruje, że neuronauka i~uczenie maszynowe są sobie bliższe, niż mogłoby się to wydawać.
Mózgowe sieci funkcjonalne okazują się mieć budowę modularną \cite{bullmore2009complex}, a~ta z~kolei ma charakter hierarchiczny \cite{meunier2009hierarchical}.
Uczenie głębokie również zaczyna iść w~kierunku budowania systemów modularnych \cite{schmidt1998modularity, hoverstad2011noise}.

Z biegiem czasu pojawiają się próby jeszcze ściślejszego zjednoczenia działania uczenia głębokiego i~działania mózgu w~neuronauce.
Wiele badaczy odnajduje bardzo szczegółowe i~konkretne przykłady analogii czy nawet podobieństwa działań technik wykorzystywanych w~uczeniu maszynowym z~działaniem zaobserwowanym wewnątrz mózgu \cite{galtier2013biological, o2014goal, o2014learning, rodriguez2004derivation, rolls2013mechanisms}.
Inni starają się zebrać je razem i~wyprowadzić teorię działania funkcjonalnego mózgu bazującą na terminologii uczenia głębokiego \cite{marblestone2016toward, de2018deep}.
W wyniku tego powstają hipotezy ugruntowane w~danych empirycznych i~podparte wieloma badaniami, ale nadal wymagające potwierdzenia i~przetestowania.

Nawet jeśli aktualnie postawione hipotezy okażą się błędne -- częściowo lub całkowicie -- nie oznacza to końca spekulacji o~możliwej integracji uczenia maszynowego i~neuronauki.
Istniejące i~wspomniane badania oraz sukcesy przy próbach wzajemnego inspirowania się tych dwóch dziedzin pokazują, że jakieś powiązanie, jakaś relacja istnieje.
Najprawdopodobniej powstaną kolejne hipotezy, ugruntowane nieco inaczej, ale z~podobnym celem i~wnioskami.

Jeśli natomiast udałoby się takowe potwierdzić, dziedziny uczenia maszynowego oraz neuronauki ponownie ściśle zostałyby ze sobą związane poprzez analogię uczenia głębokiego i~działania mózgu.
Pojawiająca się argumentacja mówi, że w~wyniku takiej analizy uczenie głębokie będzie mogło wykorzystać miliony lat ewolucji mózgu i~wzorować się na jego wewnętrznych rozwiązaniach problemów, które aktualnie nadal są poza naszymi możliwościami.
Jest bowiem niemal pewne, że mózg wykonuje wiele czynności poznawczych w~sposób znacznie bardziej optymalny, niż ma to miejsce w~uczeniu głębokim.

W przeciwnym kierunku sytuacja nie jest tak oczywista.
Eugene Izhikevich stwierdził, że ``nie można wnieść znaczącego wkładu do neuronauki poprzez symulację jednej sekundy modelu, nawet jeśli ma on rozmiary ludzkiego mózgu'' \cite{Izhikevich2006why}.
Wynika to również z~faktu, że czysto oddolna analiza ma problemy natury interpretacyjnej \cite{jonas2017could}.
Ale powiązanie neuronauki z~uczeniem głębokim daje odgórną ramę teoretyczną pozwalającą najpierw zająć się zasadami i~strukturami wyższego rzędu, a~dopiero następnie przechodzić do struktur niższego poziomu.

Być może również udana próba ściślejszej integracji uczenia maszynowego oraz działania mózgu może pomóc w~aspektach fenomenologicznych.
Pojawi się grunt dla porównania \emph{umysłu} bazowanego na mózgu z~sztucznymi systemami inteligentnymi opartymi na uczeniu głębokim, a~co za tym idzie, podstawy do odpowiadania na pytania jak ``czy maszyna może być inteligenta'', a~także ``czy maszyna może być świadoma'' \cite{mallakin2019integration}.
