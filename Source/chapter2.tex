\chapter{Modularny system symboliczny jako model działania mózgu}

% \section{Modelowanie mózgu}

\section{Kognitywistyczne podstawy i rozwój modelu systemu symbolicznego}
\label{cognitive-basics}
% file:///C:/Users/mmorus/OneDrive/Moje/Studia/Kognitywistyka/Licencjat/References/Computational%20Foundations%20of%20Natural%20Intelligence.pdf

\subsection{Koneksjonizm}

Jedną z pierwszych teorii wiązanych z mózgiem jako systemem symbolicznym był koneksjonizm.
Wyjaśnia on poznanie jako proces wynurzający się ze wzajemnego działania i współpracy pomiędzy podstawowymi elementami przetwarzającymi \cite{bechtel1993case}.

Jest on istotny w ramach omawianego aspektu, gdyż był podstawą teoretyczną abstrakcji działania mózgowych połączeń neuronalnych, która finalnie doprowadziła do pierwszych modeli sztucznych sieci neuronowych.
Jeszcze sam Alan Turing proponował utworzenie architektury komputerowej opartej na jednostki podobne neuronom \cite{copeland1996alan}.

Koneksjonistyczna idea działania poznania sugeruje, że taka abstrakcja -- mimo, że nie oddaje szczegółów działania biologicznego mózgu -- utrzymuje naturę działania mózgu, funkcje poznawcze.
Jak jednak zostało już opisane i zaznaczone w rozdziale \hyperref[chapter1]{rozdziale \ref{chapter1}}, szybko okazało się, że takie powiązanie okazuje się być znacznie bardziej skomplikowane i złożone.

\subsection{Kognitywizm}

Na koneksjonistyczne problemy modelowania działania mózgu odpowiada kognitywizm.
W tym podejściu poznanie definiuje się jako manipulację symbolami formalnymi, gdzie wnioskowanie opiera się na manipulacji reprezentacjami symbolicznymi odpowiadającymi informacji o środowisku otrzymanej przez percepcję.
Formalizacją takiego ujęcia była hipoteza systemu symboli fizycznych \cite{newell2007computer}, w której to taki system symboli fizycznych jest jednocześnie niezbędny oraz wystarczający dla inteligentnego zachowania.

Takie podejście okazało się być bardziej zbliżone działaniu mózgu, pomogło w pierwszej fazie uformować się kognitywistyce oraz dało początek niektórym sztucznym architekturom kognitywnym opartym na regułach, takie jak SOAR \cite{laird2019soar}\footnotemark.
\footnotetext{
	Projekt SOAR jest do teraz aktywnie rozwijany.
	Szczegółowe informacje na jego temat znajdują się na oficjalnej stronie: \url{https://soar.eecs.umich.edu}.
	Repozytoria związane z projektem są otwartoźródłowe i znajdują się w serwisie GitHub: \url{https://github.com/SoarGroup}.
}
Ponownie jednak taka architektura okazała się nie oddawać w pełni złożoności działania poznania u człowieka.

\subsection{Modelowanie probabilistyczne}

W ludzkim poznaniu, a co za tym idzie w modelu działania mózgu, informacje czy samo wnioskowanie nie opiera się wyłącznie na Boole'owskiej prawdzie i fałszu, tak, jak ma to miejsce w logice pierwszego stopnia, na której oparte były poprzednie architektury.
Zazwyczaj człowiek nie jest ``pewny'', ale ``przekonany''.
Umysł operuje na zdarzeniach natury indukcyjnej i pewność nigdy nie jest rzeczywiście niepodważalna.
Nie brnąc dalej w filozoficzną tematykę czym jest prawda i fałsz, wydaje się być oczywistym, że człowiek działa raczej na zasadzie prawdopodobieństw i nieokreśloności.

Matematyka jednak również była tutaj w stanie pomóc, jako, że logika probabilistyczna była już dostępna i dobrze rozwinięta.
Architektury modelujące oparte na kwantyfikacji stanu świata w terminach stopnia prawdziwości nazywanej teorią probabilistyczną \cite{jaynes1988does}, zbliżyły się jeszcze bardziej do poznania człowieka i modelowania działania ludzkiego mózgu.
Takie podejście jest podstawą wyjaśniania podejmowania optymalnych decyzji wykonywanych w warunkach niepewności.
Są postulaty wskazujące na to, że na poziomie algorytmicznym mózgi wykorzystują algorytmy wnioskowania przybliżonego (\emph{approximate inference}) \cite{andrieu2003introduction} tak, aby wytwarzać wystarczająco dobre rozwiązania dla szybkiego i oszczędnego podejmowania decyzji.

\subsection{Oddolna emergencja a odgórna abstrakcja}

Jeśli zaczniemy rozważania od odwrotnej strony przy pomocy pytania: ``jaki system sztucznie inteligenty byłby najbliższy działania rzeczywistego, biologicznego mózgu'' szybko dotrzemy do wniosku, że byłaby to jego symulacja.
Taki system oparty by był na \emph{oddolnej emergencji} (\emph{bottom-up emergence}), gdzie zachowania i poznanie symulowanego mózgu wyłoniłoby się z symulacji najprostszych jednostek, neuronów oraz ich środowiska.
Jednak jak twierdzi badacz, który zajmował się symulacją systemów wzgórzowo-korowych u ssaków \cite{izhikevich2008large} Eugene Izhikevich, nawet symulacja sekundy działania całego ludzkiego mózgu nie przyczyniłaby się znacząco do neuronauki \cite{Izhikevich2006why}.
Dodając do tego ogrom skali zadania, jakim jest symulacja ludzkiego mózgu można zrozumieć, dlaczego zwrócono się w stronę alternatywnego podejścia.

Rozwój podejść z kategorii \emph{odgórnej abstrakcji} (\emph{top-down abstraction}) jest obrazowany przez poprzednie tematy z \hyperref[cognitive-basics]{tej sekcji}.
Te algorytmy są obliczeniowo wykonywalne i dzięki temu mają potencjał użytkowy do analiz teoretycznych.
Obiecującym punktem powstałym na wyjściu implementacji założeń teoretycznych w praktyce są sztuczne sieci neuronowe, podstawa nowoczesnego głębokiego uczenia i dużej porcji całego uczenia maszynowego.
Można je wiązać z poziomem biofizycznym -- głównym źródłem porównań przy oddolnej emergencji -- jako abstrakcje biologicznych sieci neuronów opartych na częstotliwości (\emph{rate-based}) \cite{dayan2001theoretical}.
Architektury sieci mogą być wytwarzane w procesie ewolucyjnym, czy algorytmów bazowanych na strategiach ewolucyjnych \cite{real2017large}.
Są zgodne również z modelowaniem probabilistycznym, jak zauważono, algorytmy obliczeniowe powstałe w wyniku uczenia są zbliżone efektywnie do wnioskowania bayesowskiego \cite{gal2016dropout, mandt2017stochastic}.
Są związane nawet z bazą komputacjonistycznego podejścia poprzez ugruntowanie reprezentacji symbolicznych w stanach sensorycznych świata rzeczywistego \cite{harnad1990symbol}.

Użycie koneksjonistycznych sztucznych sieci neuronowych wzbogaconych o architektury i struktury innych modelów systemów przetwarzania symbolicznego wydaje się być najbliżej potencjalnej inteligencji sztucznej.
Nie jest to jednak jednoznaczne z tym, że są także najbliżej odwzorowywania działania mózgu.
Aby móc wyciągać wnioski choćby zbliżone do takowych należy wykazać także podobieństwo w przeciwnym kierunku.

\section{Zestawienie modularności mózgu z modularnością sztucznych systemów symbolicznych}


